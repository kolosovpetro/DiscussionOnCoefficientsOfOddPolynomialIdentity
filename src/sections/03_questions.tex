\begin{question}
    Although, a proof of combinatorial identity~\eqref{eq:combinatorial-identity} is already present, it is good
    to point out literature or more context on it.
    Reference to a book or article with deeper discussion.
\end{question}
\begin{question}
    Are these coefficients $\coeffA{m}{r}$ appear in widely-known mathematical literature?
\end{question}
\begin{question}
    \label{eq:question_2}
    I have struggle to understand the equation~\eqref{eq:question_1}, it takes the coefficient of $n^{2m+1}$ meaning that
    we substitute $r=m$ into~\eqref{eq:main_relation} evaluating it, if I understand it properly.
    So that coefficient of $n^{2m+1}$ is
    \begin{align*}
        \coeffA{m}{m} \frac{1}{(2m+1) \binom{2m}{m}} + \sum_{r=0}^{m} \coeffA{m}{r} \frac{(-1)^r}{r-m} \binom{r}{2m+1} \bernoulli{2r-2m} - 1
    \end{align*}
    It implies that following sum is zero
    \begin{align*}
         \sum_{r=0}^{m} \coeffA{m}{r} \frac{(-1)^r}{r-m} \binom{r}{2m+1} \bernoulli{2r-2m} = 0
    \end{align*}
    So that
    \begin{align*}
        \coeffA{m}{m} \frac{1}{(2m+1) \binom{2m}{m}} = 1; \quad \quad \coeffA{m}{m} = (2m+1) \binom{2m}{m}
    \end{align*}
    Which is indeed true because $\binom{r}{2m+1} = 0$ as $r$ runs over $0 \leq r \leq m$.
\end{question}
\begin{question}
    Almost the same problem with equation~\eqref{eq:question_2},
    taking the coefficient of $n^{2d+1}$ for an integer $d$ in the range $\frac{m}{2} \leq d \leq m-1$, we get
    \begin{align*}
        \coeffA{m}{d} = 0
    \end{align*}
    Let be $r=d$ and $k=d$ in~\eqref{eq:main_relation}, then the coefficient of $n^{2d+1}$ is
    \begin{align*}
        \coeffA{m}{d} \frac{1}{(2d+1) \binom{2d}{d}} + \sum_{r=0}^{m} \coeffA{m}{r} \frac{(-1)^r}{r-d} \binom{r}{2d+1} \bernoulli{2r-2d} - 0
    \end{align*}
    The sum
    \begin{align*}
        \sum_{r} \coeffA{m}{r} \frac{(-1)^r}{r-d} \binom{r}{2d+1} \bernoulli{2r-2d} = 0
    \end{align*}
    because $\binom{r}{2d+1}=0$ for all $r$ such that $0 \leq r \leq m$ and $d$ such that $\frac{m}{2} \leq d \leq m-1$.
\end{question}
To summarize, the value of $d$ should be in the range $d \leq \frac{m}{2}-1$ so that binomial coefficient $\binom{r}{2d+1}$
is non-zero.
For example, let be $r=m$ and $d=\frac{m}{2}-1$ then $\binom{r}{2d+1} = \binom{m}{m-1} \neq 0$ and so on for each $d \leq \frac{m}{2}-1$.
