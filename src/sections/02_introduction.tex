Assuming that following odd power identity holds
\begin{align}
    \label{eq:odd-power-identity}
    n^{2m+1} = \sum_{r=0}^{m} \sum_{k=1}^{n} \coeffA{m}{r} k^r (n-k)^r
\end{align}
Our main goal is to identify the set of coefficients $\coeffA{m}{0}, \coeffA{m}{1}, \ldots, \coeffA{m}{m}$
such that identity above is true.

Although, the recurrence relation is already given at~\cite{alekseyev2018mathoverflow}, a few key points in proof
are worth to discuss additionally.

The main idea of Alekseyev's approach was to utilize dynamic programming methods to evaluate the $\coeffA{m}{r}$ recursively,
taking the base case $\coeffA{m}{m}$ and then evaluating the next coefficient $\coeffA{m}{m-1}$
by using backtracking, continuing similarly up to $\coeffA{m}{0}$.

By applying Binomial theorem $(n-k)^r=\sum_{t=0}^{r} (-1)^t \binom{r}{t} n^{r-t} k^t$ and Faulhaber's formula
$\sum_{k=1}^{n} k^{p} = \left[ \frac{1}{p+1}\sum_{j} \binom{p+1}{j} \bernoulli{j} n^{p+1-j} \right] - \bernoulli{p+1}$, we get
\begin{align*}
    &\sum_{k=1}^{n} k^{r} (n-k)^{r}
    = \sum_{t=0}^{r} (-1)^t \binom{r}{t} n^{r-t} \sum_{k=1}^{n} k^{t+r} \\
    &= \sum_{t=0}^{r} (-1)^t \binom{r}{t} n^{r-t} \left[ \frac{1}{t+r+1} \sum_{j} \binom{t+r+1}{j} \bernoulli{j} n^{t+r+1-j} - \bernoulli{t+r+1} \right] \\
    &= \sum_{t=0}^{r} \binom{r}{t} \left[ \frac{(-1)^t}{t+r+1} \sum_{j} \binom{t+r+1}{j} \bernoulli{j} n^{2r+1-j} - \bernoulli{t+r+1} n^{r-t} \right] \\
    &= \left[ \sum_{t=0}^{r} \binom{r}{t} \frac{(-1)^t}{t+r+1} \sum_{j} \binom{t+r+1}{j} \bernoulli{j} n^{2r+1-j}  \right]
    - \left[ \sum_{t=0}^{r} \binom{r}{t} \frac{(-1)^t}{t+r+1} \bernoulli{t+r+1} n^{r-t} \right] \\
    &= \left[ \sum_{j, t} \binom{r}{t} \frac{(-1)^t}{t+r+1} \binom{t+r+1}{j} \bernoulli{j} n^{2r+1-j}  \right]
    - \left[ \sum_{t} \binom{r}{t} \frac{(-1)^t}{t+r+1} \bernoulli{t+r+1} n^{r-t} \right]
\end{align*}
Rearranging terms yields
\begin{equation}
    \left[ \sum_{j} \bernoulli{j} n^{2r+1-j} \sum_{t} \binom{r}{t} \frac{(-1)^t}{t+r+1} \binom{t+r+1}{j}  \right]
    - \left[ \sum_{t} \binom{r}{t} \frac{(-1)^t}{t+r+1} \bernoulli{t+r+1} n^{r-t} \right]
    \label{eq:rearranging-terms}
\end{equation}
We can notice that
\begin{equation}
    \label{eq:combinatorial-identity}
    \sum_{t} \binom{r}{t} \frac{(-1)^t}{r+t+1} \binom{r+t+1}{j}
    =\begin{cases}
         \frac{1}{(2r+1) \binom{2r}r} & \text{if } j=0\\
         \frac{(-1)^r}{j} \binom{r}{2r-j+1} & \text{if } j>0
    \end{cases}
\end{equation}
An elegant proof of the binomial identity~\eqref{eq:combinatorial-identity} is presented in~\cite{scheuer2023mathstackexchange}.

In particular, equation~\eqref{eq:combinatorial-identity} is zero for $0< t \leq j$.
By using~\eqref{eq:combinatorial-identity}, we are able to move $j=0$ out of summation
in~\eqref{eq:rearranging-terms} which yields
\begin{equation*}
    \begin{split}
        \sum_{k=1}^{n} k^{r} (n-k)^{r}
        &= \frac{1}{(2r+1) \binom{2r}r} n^{2r+1} + \left[ \sum_{j \geq 1} \bernoulli{j} n^{2r+1-j} \sum_{t} \binom{r}{t} \frac{(-1)^t}{t+r+1} \binom{t+r+1}{j} \right] \\
        &- \left[ \sum_{t=0}^{r} \binom{r}{t} \frac{(-1)^t}{t+r+1} \bernoulli{t+r+1} n^{r-t} \right]
    \end{split}
\end{equation*}
Simplifying above equation by using~\eqref{eq:combinatorial-identity} yields
\begin{equation*}
    \begin{split}
        \sum_{k=1}^{n} k^{r} (n-k)^{r}
        &= \frac{1}{(2r+1) \binom{2r}r} n^{2r+1}
        + \underbrace{\left[ \sum_{j \geq 1} \frac{(-1)^r}{j} \binom{r}{2r-j+1} \bernoulli{j} n^{2r-j+1} \right]}_{(\star)} \\
        &- \underbrace{\left[ \sum_{t=0}^{r} \binom{r}{t} \frac{(-1)^t}{t+r+1} \bernoulli{t+r+1} n^{r-t} \right]}_{(\diamond)}
    \end{split}
\end{equation*}
Hence, introducing $\ell=2r-j+1$ to $(\star)$ and $\ell=r-t$ to $(\diamond)$
we collapse the common terms
\begin{equation*}
    \begin{split}
        \sum_{k=1}^{n} k^{r} (n-k)^{r}
        &= \frac{1}{(2r+1) \binom{2r}r} n^{2r+1}
        + \left[ \sum_{\ell} \frac{(-1)^r}{2r+1-\ell} \binom{r}{\ell} \bernoulli{2r+1-\ell} n^{\ell} \right] \\
        &- \left[ \sum_{\ell} \binom{r}{\ell} \frac{(-1)^{r-\ell}}{2r+1-\ell} \bernoulli{2r+1-\ell} n^{\ell} \right]\\
        &= \frac{1}{(2r+1) \binom{2r}r} n^{2r+1} + 2 \sum_{\mathrm{odd \; \ell}} \frac{(-1)^r}{2r+1-\ell} \binom{r}{\ell} \bernoulli{2r+1-\ell} n^{\ell}
    \end{split}
\end{equation*}
Assuming that $\coeffA{m}{r}$ is defined by~\eqref{eq:odd-power-identity},
we obtain the following relation for polynomials in $n$
\begin{equation*}
    \sum_{r} \coeffA{m}{r} \frac{1}{(2r+1) \binom{2r}r} n^{2r+1}
    + 2 \sum_{r, \; \mathrm{odd \; \ell}} \coeffA{m}{r} \frac{(-1)^r}{2r+1-\ell} \binom{r}{\ell} \bernoulli{2r+1-\ell} n^{\ell}
    \equiv n^{2m+1}
\end{equation*}
Replacing odd $\ell$ by $k$ we get
\begin{align}
    \label{eq:main_relation}
    \sum_{r} \coeffA{m}{r} \frac{1}{(2r+1) \binom{2r}r} n^{2r+1}
    + 2 \sum_{r, \; k} \coeffA{m}{r} \frac{(-1)^r}{2r-2k} \binom{r}{2k+1} \bernoulli{2r-2k} n^{2k+1} \equiv n^{2m+1}
\end{align}
Taking the coefficient of $n^{2m+1}$ we get
\begin{align}
    \label{eq:question_1}
    \coeffA{m}{m} = (2m+1)\binom{2m}{m}
\end{align}

Taking the coefficient of $n^{2d+1}$ for an integer $d$ in the range $\frac{m}{2} \leq d < m$, we get
\begin{align}
    \label{eq:question_2}
    \coeffA{m}{d} = 0
\end{align}

Taking the coefficient of $n^{2d+1}$ for $d$ in the range $\frac{m}{4} \leq d < \frac{m}{2}$ we get
\begin{align}
    \label{eq:question_3}
    \coeffA{m}{d} \frac{1}{(2d+1) \binom{2d}{d}}
    +2 (2m+1) \binom{2m}{m}\binom{m}{2d+1} \frac{(-1)^m}{2m-2d} \bernoulli{2m-2d} = 0
\end{align}
i.e
\begin{equation*}
    \coeffA{m}{d} = (-1)^{m-1} \frac{(2m+1)!}{d!d!m!(m-2d-1)!} \frac{1}{m-d} \bernoulli{2m-2d}
\end{equation*}
Continue similarly we can compute $\coeffA{m}{r}$ for each integer $r$ in range $\frac{m}{2^{s+1}} \leq r < \frac{m}{2^{s}}$,
iterating consecutively over $s=1,2,\ldots$ by using previously determined values of $\coeffA{m}{d}$ as follows
\begin{equation*}
    \coeffA{m}{r} =
    (2r+1) \binom{2r}{r} \sum_{d \geq 2r+1}^{m} \coeffA{m}{d} \binom{d}{2r+1} \frac{(-1)^{d-1}}{d-r}
    \bernoulli{2d-2r}
\end{equation*}
Finally, we are capable to define the following recurrence relation for coefficient $\coeffA{m}{r}$
\begin{definition} (Definition of coefficient $\coeffA{m}{r}$.)
    \begin{equation}
        \label{eq:definition_coefficient_a}
        \coeffA{m}{r} =
        \begin{cases}
        (2r+1)
            \binom{2r}{r} & \mathrm{if} \; r=m \\
            (2r+1) \binom{2r}{r} \sum_{d \geq 2r+1}^{m} \coeffA{m}{d} \binom{d}{2r+1} \frac{(-1)^{d-1}}{d-r}
            \bernoulli{2d-2r} & \mathrm{if} \; 0 \leq r<m \\
            0 & \mathrm{if} \; r<0 \; \mathrm{or} \; r>m
        \end{cases}
    \end{equation}
\end{definition}
where $\bernoulli{t}$ are Bernoulli numbers~\cite{bateman1953higher}.
It is assumed that $\bernoulli{1}=\frac{1}{2}$.

For example,
\begin{table}[H]
    \begin{center}
        \begin{tabular}{c|cccccccc}
            $m/r$ & 0 & 1       & 2      & 3      & 4   & 5    & 6     & 7     \\
            \hline
            0     & 1 &         &        &        &     &      &       &       \\
            1     & 1 & 6       &        &        &     &      &       &       \\
            2     & 1 & 0       & 30     &        &     &      &       &       \\
            3     & 1 & -14     & 0      & 140    &     &      &       &       \\
            4     & 1 & -120    & 0      & 0      & 630 &      &       &       \\
            5     & 1 & -1386   & 660    & 0      & 0   & 2772 &       &       \\
            6     & 1 & -21840  & 18018  & 0      & 0   & 0    & 12012 &       \\
            7     & 1 & -450054 & 491400 & -60060 & 0   & 0    & 0     & 51480
        \end{tabular}
    \end{center}
    \caption{Coefficients $\coeffA{m}{r}$, sequences A302971 and A304042 in~\cite{OEIS}.}
    \label{tab:table_of_coefficients_a}
\end{table}


Properties of the coefficients $\coeffA{m}{r}$
\begin{itemize}
    \item $\coeffA{m}{m} = \binom{2m}{m}$
    \item $\coeffA{m}{r} = 0$ for $m < 0$ and $r > m$
    \item $\coeffA{m}{r} = 0$ for $r < 0$
    \item $\coeffA{m}{r} = 0$ for $\frac{m}{2} \leq r < m$
    \item $\coeffA{m}{0} = 1$ for $m \geq 0$
    \item $\coeffA{m}{r}$ are integers for $m \leq 11$
    \item Row sums: $\sum_{r=0}^{m} \coeffA{m}{r} = 2^{2m+1} - 1$
\end{itemize}
