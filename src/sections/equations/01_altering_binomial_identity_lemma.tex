\begin{lemma}[Piecewise Binomial identity]
    \label{lem:piecewise-binomial-identity}
    For integers $r, j$, we have
    \begin{align*}
        \sum_{t} \binom{r}{t} \frac{(-1)^t}{r+t+1} \binom{r+t+1}{j}
        = \begin{cases}
              \displaystyle \frac{1}{(2r+1) \binom{2r}{r}} & \text{if } j=0 \\[0.8em]
              \displaystyle \frac{(-1)^r}{j} \binom{r}{2r-j+1} & \text{if } j>0
        \end{cases}
    \end{align*}
    \begin{proof}
        For $j=0$ we have
        \begin{align*}
            \sum_t \binom{r}{t} \frac{(-1)^t}{r+t+1} = \sum_t \binom{r}{t} (-1)^t \int_0^1 z^{r+t} \, dz
        \end{align*}
        Because $\frac{1}{r+t+1} = \int_0^1 z^{r+t} \, dz$.
        \begin{align*}
            \sum_t \binom{r}{t} (-1)^t \int_0^1 z^{r+t} \, dz
            = \int_0^1 z^r \left( \sum_t \binom{r}{t} (-1)^t z^{t} \right) \, dz
            = \int_0^1 z^r (1 - z)^r \, dz
        \end{align*}
        The work~\cite{sury2004identities} provides the identity $\binom{n}{k}^{-1}=(n+1)\int_0^1 z^k(1-z)^{n-k}\,dz$.
        By setting $n=2r$ and $k=r$ yields
        \begin{align*}
            \sum_t \binom{r}{t} \frac{(-1)^t}{r+t+1} = \int_0^1 z^r (1-z)^{r}\,dz = \binom{2r}{r}^{-1} \frac{1}{2r+1}
        \end{align*}
        This completes the proof for $j=0$.

        For $j > 0$
        \begin{align*}
            \sum_t \binom{r}{t} \frac{(-1)^t}{r+t+1} \binom{r+t+1}{j} =  \sum_t \frac{(-1)^t}{j} \binom{r}{t} \binom{r+t}{j-1}
        \end{align*}
        Because $\binom{n}{k} = \frac{n}{k} \binom{n-1}{k-1}$.
        Now apply the coefficient extraction $[z^k]$ to represent the coefficient of $z^k$.
        For example: $[z^k] (1+z)^r = \binom{r}{k}$.
        Therefore,
        \begin{align*}
            \sum_t \frac{(-1)^t}{j}  \binom{r}{t} \binom{r+t}{j-1}
            = \sum_t \frac{(-1)^t}{j} \binom{r}{t} [z^{j-1}] (1+z)^{r+t}
            = [z^{j-1}] \sum_t \frac{(-1)^t}{j} \binom{r}{t} (1+z)^{r+t}
        \end{align*}
        By factoring out $(1+z)^r$ from the sum
        \begin{align*}
            [z^{j-1}] \sum_t \frac{(-1)^t}{j} \binom{r}{t} (1+z)^{r+t}
            &= [z^{j-1}] (1+z)^r \sum_t \frac{(-1)^t}{j} \binom{r}{t} (1+z)^t
        \end{align*}
        Now apply the binomial theorem to the inner sum
        \begin{align*}
            \sum_t \binom{r}{t} (-1)^t (1+z)^t = \left(1 - (1+z)\right)^r = (-z)^r = (-1)^r z^r
        \end{align*}
        Hence, for $j>0$
        \begin{align*}
            \sum_t \binom{r}{t} \frac{(-1)^t}{r+t+1} \binom{r+t+1}{j} = \frac{(-1)^r}{j} [z^{j-1}] (1+z)^r z^r
        \end{align*}
        By applying the identity $[z^{p-q}]A(z)=[z^p]z^qA(z)$
        \begin{align*}
            \frac{(-1)^r}{j}  [z^{j-1}] (1+z)^r z^r = \frac{(-1)^r}{j}  [z^{j-1-r}] (1+z)^r = \frac{(-1)^r}{j} \binom{r}{j - 1 - r}
        \end{align*}
        Finally, we use the symmetry $\binom{n}{k} = \binom{n}{n-k}$ to show that for $j > 0$
        \begin{align*}
            \sum_t \binom{r}{t} \frac{(-1)^t}{r+t+1} \binom{r+t+1}{j} = \frac{(-1)^r}{j} \binom{r}{j - 1 - r} = \frac{(-1)^r}{j} \binom{r}{2r - j + 1}
        \end{align*}
%        Thus, for $j > 0$
%        \begin{align*}
%            \sum_t \binom{r}{t} \frac{(-1)^t}{r+t+1} \binom{r+t+1}{j}
%            = \frac{(-1)^r}{j} \binom{r}{2r - j + 1}
%        \end{align*}
        This completes the proof.
    \end{proof}
\end{lemma}

