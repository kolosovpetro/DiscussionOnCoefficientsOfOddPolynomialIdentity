Interestingly enough that the odd power identity above is a Pascal-type identity
in terms of bivariate function $k(n-k)$ and numbers $\coeffA{m}{r}$.
We may see it by comparing the Pascal's identity itself~\cite{macmillan2011proofs}
\begin{align*}
(n+1)
    ^{m+1}-1 = \sum_{r=0}^{m} \binom{m+1}{r} \left( 1^{r}+2^{r} + \dots + n^{r} \right)
\end{align*}
with identity in terms of bivariate function $k(n-k)$ and numbers $\coeffA{m}{r}$
\begin{corollary}[Bivariate Pascal's identity]
    For integers $n \geq 1$ and $m\geq 0$
    \begin{align*}
    (n+1)
        ^{2m+1} - 1
        &= \sum_{r=0}^{m} \coeffA{m}{r} \left[ 1^r n^r + 2^r (n-1)^r + 3^r (n-2)^r + 4^r (n-3)^r + \cdots + n^r (n+1-n)^r  \right] \\
        &= \sum_{r=0}^{m} \coeffA{m}{r} \left[ n^r + (2n-2)^r + (3n-6)^r + (4n-12)^r + \cdots + n^r  \right] \\
        &= \sum_{r=0}^{m} \coeffA{m}{r} \sum_{k=1}^{n} (k (n+1-k))^r
    \end{align*}
\end{corollary}
\begin{definition}[Bivariate sum $T_m$]
    For integers $n,k$ and $m \geq 0$
    \label{def:bivariate-sum-Tm}
    \begin{align*}
        T_{m}(n,k) = \sum_{r=0}^{m} \coeffA{m}{r} k^r (n-k)^r
    \end{align*}
\end{definition}

\begin{proposition}[Symmetry of $T_m$]
    \label{prop:Tm-symmetry}
    For integers $n,k$ and $m\geq 0$
    \begin{align*}
        T_{m} (n, k) = T_{m} (n, n-k)
    \end{align*}
\end{proposition}
\begin{proposition}[Forward Recurrence for $T_m$]
    For integers $n,k$ and $m\geq 0$
    \label{prop:Tm-recurrence-forward}
    \begin{align*}
        T_{m} (n,k) = \sum_{t=1}^{m+1} (-1)^{t+1} \binom{m+1}{t} T_{m} (n+t, k)
    \end{align*}
    \begin{proof}
        The polynomial $T_{m} (n,k)$ is a polynomial of degree $m$ in $n$.
        Thus, the forward difference with respect to $n$ is
        $\Delta^{m+1} T_{m} (n, k) = \sum_{t=0}^{m+1} (-1)^{t} \binom{m+1}{t} T_{m} (n+t, k) = 0$.
        By isolating $(-1)^{0} \binom{m+1}{0} T_{m} (n-0, k)$ yields
        $T_{m} (n, k) = (-1) \sum_{t=1}^{m+1} (-1)^{t} \binom{m+1}{t} T_{m} (n+t, k)$.
    \end{proof}
\end{proposition}

\begin{proposition}[Odd power forward decomposition]
    \label{prop:odd-power-decomposition-forward}
    For non-negative integers $m$ and $n$
    \begin{align*}
        n^{2m+1} = \sum_{k=1}^{n} \sum_{t=1}^{m+1} (-1)^{t+1} \binom{m+1}{t} T_{m} (n+t, k)
    \end{align*}
    \begin{proof}
        Direct consequence of~\eqref{prop:odd-power-identity}
        and forward recurrence~\eqref{prop:Tm-recurrence-forward}.
    \end{proof}
\end{proposition}
For example:
\begin{itemize}
    \item $3^5 = \binom{3}{1} 1023 - \binom{3}{2} 2643 + \binom{3}{3} 5103$
    \item $3^5 = \binom{4}{1} 1023 - \binom{4}{2} 2643+ \binom{4}{3} 5103 - \binom{4}{4} 8403$
\end{itemize}
\begin{proposition}[Forward Recurrence for $T_m$ multifold]
    For integers $m \geq 0, \; n, \; k$ and $s \geq 1$
    \label{prop:Tm-recurrence-forward-multifold}
    \begin{align*}
        T_{m} (n,k) = \sum_{t=1}^{m+s} (-1)^{t+1} \binom{m+s}{t} T_{m} (n+t, k)
    \end{align*}
\end{proposition}

\begin{proposition}[Odd power forward decomposition multifold]
    \label{prop:odd-power-decomposition-forward-multifold}
    For non-negative integers $m$, $n$ and $s \geq 1$
    \begin{align*}
        n^{2m+1} = \sum_{k=1}^{n} \sum_{t=1}^{m+s} (-1)^{t+1} \binom{m+s}{t} T_{m} (n+t, k)
    \end{align*}
    \begin{proof}
        Direct consequence of~\eqref{prop:odd-power-identity}
        and forward recurrence multifold~\eqref{prop:Tm-recurrence-forward-multifold}.
    \end{proof}
\end{proposition}
For example:
\begin{itemize}
    \item $2^3 = \binom{5}{1} 26 - \binom{5}{2}44 + \binom{5}{3} 62 - \binom{5}{4}80+\binom{5}{5}98$
    \item $5^3 = \binom{3}{1} 215 - \binom{3}{2} 305 + \binom{3}{3} 395$
\end{itemize}
\begin{proposition}[Negated binomial form]
    \label{prop:negated-binomial-form}
    For integers $n, \; a$ and $m\geq 0$ such that $n-2a \geq 0$
    \begin{align*}
    (n-2a)
        ^{2m+1} = \sum_{r=0}^{m} \sum_{k=a+1}^{n-a} \coeffA{m}{r} (k-a)^r (n-a-k)^r
    \end{align*}
    \begin{proof}
        By observing the summation limits we can see that $k$ runs as $k=a+1,a+2,a+3,\ldots,a+n-a$, which
        implies that $(k-a)=1,2,3,\ldots, n$.
        By observing the term $(n-k-a)$ we see that $(n-k-a)=n-1,n-2,n-3,\ldots,0$.
        Thus, by reindexing the sum
        $(n-2a)^{2m+1} = \sum_{r=0}^{m} \sum_{k=1}^{n-2a} \coeffA{m}{r} (a+k-a)^r (n-(a+k)-a)^r$
        the statement~\eqref{prop:negated-binomial-form} is equivalent to~\eqref{prop:odd-power-identity}
        with setting $n \rightarrow n-2a$.
    \end{proof}
\end{proposition}
